\documentclass[draft]{article}

\usepackage{caption}
\usepackage[a4paper,margin=2.5cm,footskip=.5cm]{geometry}
\usepackage{graphicx}
\usepackage{hyperref}
\usepackage{placeins}
\usepackage{subcaption}

\title{Combining Bayes and Density Tree Classifier}
\author{Constantin Pape\\
        Marcus Theisen}
\date{\today}

\begin{document}
\maketitle

\section{Introduction}

In this project we combine two different classifiers to address the issue of generating new instances for data with interdependent feature dimensions. As basis we take the Naive Bayes Classifier, that treats feature dimensions as independent and the Density Tree Classifier, that captures the dependency between the feature dimensions. 
\newline
Both Classifiers share the concept of learning the data by learning the posterior $p(B|A)$ in Bayes Theorem 
\begin{equation} \label{eq1}
	p(A|B) = \frac{p(B|A) ~ P(A)}{P(B)} .
\end{equation}
As a reference dataset we use a downsampled version of the MNIST-dataset and filter it for the numerals 3 and 8. We test the classifiers on the full dataset and a version that is reduced to the two most relevant dimensions.
For further applications we use more complex datasets: Words % TODO etc... 
\newline
The Classifiers are implemented in C++, the source code is available on GitHub: \newline
\url{https://github.com/consti123/machine_learning}

\section{Basis Classifiers}

First we give a brief summary of the two underlying classifiers and present their results on the MNIST-dataset.

\subsection{Bayes Classifier}

The Naive Bayes Classifier treats all feature dimensions as independent. It learns the posterior by creating a histogram 
for each dimension and class. The bin width (in each dimension) is chosen according to the formula
\begin{equation}
	\Delta X = \frac{2 ~ \mathrm{IQR}(X)}{N^{\frac{1}{3}}}
\end{equation}
where $X$ is the dataset, IQR the inter-quartile-range and $N = |X|$. However this 
rule might lead to too small bins for data with a small spread. To counter this effect,
we enforce a maximal bin number of $\sqrt{N}$, adjusting the bin width if necessary.
After determining the bin width the histograms are constructed from the training data and normalised.
\newline
The Bayes Classifier predicts the labels for new data by calculating the likelihood according to
\ref{eq1} for each class. The posterior $p(B|A)$ is given by the probability of the corresponding bin.
The prior $p(A)$ is given by the fraction of class instances to total instances in the training data.
Then the class with the biggest value for $p(B|A) ~ p(A)$ is predicted.
\newline
For generating the Bayes Classifier iterates over the dimensions.
For each dimension one of the 5 most probable bins is chosen with its probability and 
then it is sampled uniformly from this bin.

\subsubsection{Results}

The Bayes Classifier performs well on the full and reduced dataset, see table \ref{tab1} for details.
	
\begin{table}[h]
	\centering
	\begin{tabular}{l c}
		Data	&	Classification Rate	\\
		Full	&	0.8654				\\
	 	Reduced & 	0.8376				\\
	\end{tabular}
	\caption{Results for the Bayes Classifier}
	\label{tab1}
\end{table}

For the generation method chosen about 10 \% of the results can be identified as 3s.
The results are quite sparse, because we only sample from the 5 most probable bins.
See figure \ref{fig1} for examples.

\begin{figure}[h]
	\centering
	\begin{subfigure}[b]{0.4\textwidth}
		\includegraphics[width=\textwidth]{graphics/gen_bayes_3.png}
	\end{subfigure}
	\begin{subfigure}[b]{0.4\textwidth}
		\includegraphics[width=\textwidth]{graphics/gen_bayes_4.png}
	\end{subfigure}
	\caption{Instances generated with the BayesClassifier.}
	\label{fig1}
\end{figure}

\subsection{Density Tree Classifier} \label{dt}

The Density Tree Classifier captures the interdependence of the feature dimensions. This is done by 
building a binary tree from the data for each class.
First all data is put in the root node of the tree. This node is splitted in two children nodes according to a split criterion.
This is repeated for all nodes until a termination criterion is met.
\newline
We started with a simple splitting criterion in the beginning:
We iterate over the dimensions. For each dimension we create a set of thresholds. 
The instances to the left of this threshold (in current dimension) belong to the left node, the ones to 
the right to the right node. For each instance (except the two outer ones) two thresholds are created, one to the left
of the instance and one to the right.
Then a gain-function is calculated for each threshold. Finally the split that maximizes this gain is chosen.
We implement the criterion with gain-function
\begin{equation}
	\mathrm{gain} ~ = ~ ( \frac{N_l}{N} )^2 ~ \frac{1}{V_l} ~ +  ~ ( \frac{N_r}{N} )^2 ~ \frac{1}{V_r} 
\end{equation}
where $N$ is the total number of instances in the node, $N_l$, $N_r$ the number of instances to the left / right
and $V_l$, $V_r$ the volume to the left / right.
From now on we will refer to this criterion as the standard criterion. We will introduce different split 
criteria later.
\newline
To each node the probability $p_i = \frac{N_i}{N ~ V_i}$ is assigned. Here $N$ is the total number of instances
and $N_i$, $V_i$ the values for the corresponding node.
\newline
A node is terminated if it has either reached a maximal depth in the tree or has fallen below the minimal number
of instances $N_min = N^{\frac{1}{3}}$.
\newline
To predict the labels for new data the posterior is calculated for each class by finding the leave node in which this data point
belongs. Then the posterior is identified with the probability of this node. The prior is calculated like in the
Bayes Classifier. The class with the biggest product of posterior and prior is predicted.
\newline
New instances are generated by walking the tree: Starting at the root node the children nodes are selected with their
probability until a leaf node is reached. Then the new data point is sampled uniformly from this node.

\subsubsection{Results}\label{res}

This classifier performs considerably worse than the Bayes Classifier. For the reduced data it achieves a 
classification rate of approximately 68 \%, which is 15 \% worse than the Bayes Classifier. 
For the full dataset the classification rate is below 50 \%!

\begin{table}[h]
	\centering
	\begin{tabular}{l c}
		Data	&	Classification Rate	\\
		Full	&	0.4928				\\
	 	Reduced & 	0.6860				\\
	\end{tabular}
	\caption{Results for the Density Tree Classifier}
	\label{tab2}
\end{table}

To test the generation ability we generate 50 new instances of data.
Only in some of these 3s are barely discernible. In general all are noisy.
See \ref{fig2} for examples.

\begin{figure}[h]
	\centering
	\begin{subfigure}[b]{0.4\textwidth}
		\includegraphics[width=\textwidth]{graphics/gen_dt_1.png}
	\end{subfigure}
	\begin{subfigure}[b]{0.4\textwidth}
		\includegraphics[width=\textwidth]{graphics/gen_dt_2.png}
	\end{subfigure}
	\caption{Instances generated with the Density Tree.}
	\label{fig2}
\end{figure}

\section{Copula Classifier}

The main task of this project is to construct a classifier that exploits the ability of the Density Tree Classifier to capture the 
interdependency of the feature dimensions while keeping the prediction performance of the Bayes Classifier.
This way we hope to increase the generation power compared to Bayes Classifier and Density Tree Classifier.
\newline
This is achieved by combining both classifiers: We use the fact that the cumulative density function (CDF)
can be calculated from the (normalised) histograms of the Bayes Classifier.
This can be combined with a Density Tree Classifier that is trained of the rank order transformation of
the training data (result is called copula).
% TODO explain copula
\newline
The training of the new classifier can be summarized as follows:
\begin{enumerate}
\item Train the Bayes Classifier on the training data.
\item Compute the copula of the training data.
\item Train the Density Tree on the copula.
\item Compute the CDF of the histograms of the Bayes Classifier. 
\end{enumerate}
Then labels can be predicted for new data via the following steps:
\begin{enumerate}
\item The posterior is calculated with the Bayes Classifier. 
\item The CDF is applied to the data to transform it to copula space.
\item The posterior of the Density Tree is calculated on the transformed data.
\item The product of both posteriors and the prior is taken.
\end{enumerate}
This is done for each class and the class with the biggest resulting value is predicted.
\newline
New data is generated via:
\begin{enumerate}
\item Copula data is generated with the Density Tree.
\item It is mapped back with the inverse CDF from the Bayes Classifier.
\end{enumerate}
In the following we call the resulting classifier the Copula Classifier.

\subsection{Splits}

To improve on the underlying Density Tree Classifier, we implement new split criteria.
First we use the split criterion already discussed in \ref{dt}. 
Then we implement a similar criterion, which uses a different gain function:
\begin{equation}
	\textrm{gain} ~  = ~ | ~ N_l ~ V_r ~ - ~ N_r ~ V_l ~ |
\end{equation}
This criterion is supposed to yield more balanced splits than the original, because it balances the 
number of instances and the volumes of the splits. We will refer to it as alternative criterion.
\newline
Finally we implement a criterion based on the gradient of each datapoint.
For this we need an estimation of the gradient. First we find the k nearest neighbour of
the datapoint. Then we estimate the gradient as the displacement to the centre of these neighbours.
Additionally the density is estimated by the average distance of the datapoint to the neighbours.
We choose to split at the datapoint, which has the biggest value for the gradient divided by density.
Then we split along the dimension, where the gradient is biggest.

For a first comparison of the criteria we plot the distribution of the values for the gains / gradient
over the thresholds / instances. See figures \ref{fig2a}, \ref{fig2b} and \ref{fig2c}.

\begin{figure}[h]
	\centering
	\includegraphics[width=.8\textwidth]{graphics/defsplit_dimimp2.png}
	\caption{Distribution of the gain over the thresholds for the standard split.
	Split for the 2nd most important dimension, top tree level.}
	\label{fig2a}
\end{figure}

\begin{figure}[h]
	\centering
	\includegraphics[width=.8\textwidth]{graphics/altsplit_dimimp1.png}
	\caption{Distribution of the gain over the thresholds for the alternative split.
	Split for the most important dimension, top tree level.}
	\label{fig2b}
\end{figure}

\begin{figure}[h]
	\centering
	\includegraphics[width=.8\textwidth]{graphics/gradsplit_full.png}
	\caption{Distribution of the gradient over the instances for the gradient split.
	Split for the full data, top tree level.}
	\label{fig2c}
\end{figure}

One can see that the standard and alternative criterion yield a distribution  
of gains with 2 peaks (this is also true for other dimensions).
The distribution of the gradients is quite noisy and there is no clear structure discernible.

\FloatBarrier

To further examine the splits we also plot the binary tree produced for the reduced dataset.
These are shown in figures \ref{fig2d}, \ref{fig2e} and \ref{fig2f}. 
Here the label of the nodes corresponds to the number of instances assigned to them.

\begin{figure}[h]
	\centering
	\includegraphics[width=.8\textwidth]{graphics/tree_def_reduced.png}
	\caption{Tree for the standard criterion on reduced dataset.}
	\label{fig2d}
\end{figure}

\begin{figure}[h]
	\centering
	\includegraphics[width=.8\textwidth]{graphics/tree_alt_reduced.png}
	\caption{Tree for the alternative criterion on reduced dataset.}
	\label{fig2e}
\end{figure}

\begin{figure}[h]
	\centering
	\includegraphics[width=.8\textwidth]{graphics/tree_grad_reduced.png}
	\caption{Tree for the gradient criterion on reduced dataset.}
	\label{fig2f}
\end{figure}

These trees suggest, that the alternative criterion produces the most balanced splits, whereas the 
standard criterion produces the trees with the big leaves, but not too many very small leaves.
The gradient criterion seems to produce splits with many big leaves and many very small leaves.

\FloatBarrier

\subsection{Results}

To evaluate the split criteria thoroughly, we sweep the parameters they depend on.
\newline
First we sweep different tree depth values for the standard and alternative criterion.
See table \ref{tab3} for the results. The performance of the standard criterion on the full data 
is better by approximately 5 \% for the larger depth values.
However the performance for the reduced data slightly with increasing depth.
For the alternative criterion the performance decreases for full and reduced data with increasing depth.
In addition we fear that overfitting effects might be stronger for too large depth values.
Hence we continue with a maximal tree depth of 4 for the following tests.

\begin{table}[h]
	\centering
	\begin{tabular}{l c c c c}
				&   Standard  &					& 	Alternative	&				\\
		Depth	&	Full Data & Reduced Data	& 	Full Data	& Reduced Data	\\
		  4		&	0.5285	  &	0.8376			&	0.6372		&	0.8364		\\
		  8 	& 	0.5285	  & 0.8363			&	0.5286		&	0.8355		\\
		  10	&   0.5285	  & 0.8343			&	0.5285		&	0.8308		\\
		  15	& 	0.5806	  & 0.8307			&	0.5285		&	0.8299		\\
		  20	&   0.5823	  & 0.8298			&	0.5285		&	0.8299		\\
		  40	&	0.5847	  & 0.8299			&	0.5285		&	0.8299		\\
	\end{tabular}
	\caption{Results for different tree depths for the standard and alternative criterion.}
	\label{tab3}
\end{table}

Then we sweep the number of nearest neighbours k, that are taken into account for the calculation 
of gradient and density in the gradient split criterion.
\newline
The results are shown in \ref{tab4}. In contrast to the standard criterion the gradient split 
performs better on the full data. There it performs considerably better than the standard criterion.
We further see, that the quality of the splits decrease drastically with too big k-value.
We choose to continue with k = 10 for the following tests.
\newline
It should be noted that calculating this split criterion takes considerably longer than calculating
the standard or alternative criterion.

\begin{table}[h]
	\centering
	\begin{tabular}{l c c}
		  k		&	Full Data & Reduced Data	\\
		  5		&	0.8406	  &	0.8188			\\
	 	  10 	& 	0.8447	  & 0.8245			\\
		  15	&   0.8428	  & 0.8348			\\
		  30	& 	0.5357	  & 0.8159			\\
	\end{tabular}
	\caption{Results for different numbers of nearest neighbours for the gradient criterion.}
	\label{tab4}
\end{table}

Next we look at the generation of new instances with the Copula Classifier.
Figures \ref{fig3}, \ref{fig4} and \ref{fig5} show instances generated with the three
split criteria.

\begin{figure}[h]
	\centering
	\begin{subfigure}[b]{0.4\textwidth}
		\includegraphics[width=\textwidth]{graphics/gen_defsplit1.png}
	\end{subfigure}
	\begin{subfigure}[b]{0.4\textwidth}
		\includegraphics[width=\textwidth]{graphics/gen_defsplit3.png}
	\end{subfigure}
	\caption{Instances generated with the Copula Classifier (default criterion).}
	\label{fig3}
\end{figure}

\begin{figure}[h]
	\centering
	\begin{subfigure}[b]{0.4\textwidth}
		\includegraphics[width=\textwidth]{graphics/gen_altsplit1.png}
	\end{subfigure}
	\begin{subfigure}[b]{0.4\textwidth}
		\includegraphics[width=\textwidth]{graphics/gen_altsplit2.png}
	\end{subfigure}
	\caption{Instances generated with the Copula Classifier (alternative criterion).}
	\label{fig4}
\end{figure}

\begin{figure}[h]
	\centering
	\begin{subfigure}[b]{0.4\textwidth}
		\includegraphics[width=\textwidth]{graphics/gen_gradsplit2.png}
	\end{subfigure}
	\begin{subfigure}[b]{0.4\textwidth}
		\includegraphics[width=\textwidth]{graphics/gen_gradsplit1.png}
	\end{subfigure}
	\caption{Instances generated with the Copula Classifier (gradient criterion).}
	\label{fig5}
\end{figure}

\FloatBarrier

About 10 to 15 \% of the instances generated with all three criteria are discernible as 3s.
In comparison the results with the Bayes Classifier they look more noisy, but the shape of the 3 is 
in general more distinct. Qualitatively there is no difference between the generated data of the three criteria.

Next we compare all three criteria and also compare them with the Bayes Classifier.
Table \ref{tab5} shows all results. We see that the Bayes Classifier has the best classification
results on full and reduced data. The only split criterion that yields good results on the full dataset is
the gradient criterion. However it takes about 140 times longer for training than the other criteria.

\begin{table}[h]
	\centering
	\begin{tabular}{l c c c c}
		Split / Class.&	Bayes 	& Standard	& Alternative 	& Gradient	\\
		Full		&	0.8654	&	0.5285	&	0.6372		& 0.8245	\\
	 	t\_train	&   0.7 s	&	61.3 s	&	62.7 s		& 140 min	\\
		Reduced 	& 	0.8376	&	0.8375	&	0.8364		& 0.8245	\\
		t\_train	&   18.7 ms	&	1.6 s	&   1.6 s		& 573 s		\\
   		Generating	& 	0.9391 $\pm$ 0.0265	&	0.8886 $\pm$ 0.0095		& 0.8158 $\pm$ 0.0123 & 0.7998 $\pm$ 0.0126 \\
	\end{tabular}
	\caption{Results for different numbers of nearest neighbors for the gradient criterion.}
	\label{tab5}
\end{table}

We further evaluate the generating performances with the Random Forest Classifier from sklearn.
For this we train it on the complete MNIST dataset (i.e. with all numerals not only 3 and 8).
Then we predict the labels for 1000 instances generated with the different split criteria and 
the Bayes Classifier. This is repeated 50 times to exclude randomness effects from the Random Forest Classifier.
The forest is set up to consist of 30 trees.
It achieves a classification rate of 0.9223 $\pm$ 0.0007.
The classification for the generated data is reported in the last row of table \ref{tab5}. 
The errors were estimated with the standard deviation of the 50 trials.
\newline
According to this classification the Bayes Classifier generates the best data. However this might
be explained by the fact that it generates sparse images with the generation method we have chosen.
This leads to less noise in the images and might increase the correct prediction of the Random Forest Classifier 
also for generated data that is not discernible (by human standards) as 3.
To investigate this hypothesis one could try to weaken the noise in the generated instances of the Copula Classifier,
e.g. by introducing a threshold and test whether this way one obtains better generation results.
\newline
The split criterion that performs best in this validation is the standard criterion.
It is better than the other two criteria, which perform at level, by nearly 8 \%.
This is quite surprising, because this criterion performed worst with regard to prediction.
\newline
To further evaluate the generating performance of the Copula Classifier, we will turn to more complex data,
finalizing with a dataset of words. We hope that the Copula Classifier will profit from its ability to take into 
account feature dimension interdependency here.

\section{Complex Data}
Firstly let us test our classifiers against the classification problem generator from sklearn
\url{sklearn.datasets.make_classification} which produces a Madelon-like dataset.
Secondly, we will set up scheme to train our algorithm to generated valid words.
The lists of British words used are generated by scowl \url{http://wordlist.aspell.net/}.

\subsection{Madelon (sklearn)}
We will consider datasets composed of 5000 instances (4000 train, 1000 test) to check the validity
of our algorithm for a non-linear hypercube problem with two classes and 20 feature dimensions.
By incrementing the number of informative features, we can add covariance in order to
increase complexity.

\begin{table}[h]
	\centering
	\begin{tabular}{l c c c c  }
		\# informative features	& Bayes 	& Standard	& Alternative & Gradient	\\
		2			&	0.882	&	0.881	& 0.771		& 0.874 \\
	 	3			&   0.826	&	0.826	& 0.561		& 0.825 \\
		5			& 	0.801	&	0.799	& 0.664		& 0.789	\\
		10			& 	0.777	&	0.775	& 0.660		& 0.721	\\
	\end{tabular}
	\caption{Correct classification rate (accuracy) in $\%$ for different classifiers.}
	\label{tab6}
\end{table}

As seen in Table~\ref{tab6} the Bayes and Standard Copula perform equally well. This
behaviour is unexpected since the Copula should take the covariance interaction into account.
The Alternative and Gradient splits show that the choice of the split criterion
effects the accuracy but unfortunately Bayes Classifier keeps the upperhand.
\newline
For the Standard criterion the result seems to be mainly influenced by the Bayes part,
since the outcomes are almost always equal.
Therefore it might be worth considering to adjust the probability of the density tree part
by reweighting it.
Another idea to overcome the performance of Bayes would be to develop even more sophisticated
split criteria.



\subsection{Wordlists}
Lastly the Bayes, Density Tree and Copula Classifiers are trained with datasets consisting of 'valid words' and 'non-words'
of a given length. Our initial list of valid words takes into account all British-spelled words
excluding abbreviations, special signs and proper names (e.g. abbrv., we're, Africa, Einstein, ...).
Non-words are random character strings not included in those wordlists.
The goal is to subsequently generate valid words.
\newline
The number of features is determined by the word length
and the feature space has the dimension 26 according to the low-letter alphabet.
We will study the behaviour of the classifiers for words of length 5,7 and 10 and discuss the results.
Again the train set makes up $80\%$ and the test set $20\%$ of the data.
As until now the standard split performed best, we exclusively use it in the following.
\newline
Note that the setup is formally equal to a binary classification,
where true stands for valid word and false for non-word.
Hence we can analyse the errors of the classifiers accordingly.

\begin{table}[h]
	\centering
	\begin{tabular}{l c c c c c c }
		wordlength	& \multicolumn{2}{c}{Bayes} & \multicolumn{2}{c}{Copula}	& \multicolumn{2}{c}{Density Tree} 	\\
					& ACC		& GAIN			& ACC			& GAIN			& ACC			& GAIN				\\
		5			&	0.846	&	1.691		&	0.843		& 1.686			& 0.638 		& 1.368				\\
	 	7			&   0.926	&	1.852		&	0.926		& 1.853			& 0.708			& 1.416				\\
		10 			& 	0.967	&	1.935		&	0.968		& 1.935			& 0.719			& 1.438				\\
	\end{tabular}
	\caption{Correct classification rate (accuracy) and gain in accuracy.}
	\label{tab7}
\end{table}

Table~\ref{tab7} shows the classification rate and gain of each classifier tested.
As expected from earlier tests the Bayes and Copula perform equally well and the Density Tree (depth 10) lacks in
accuracy. The gain is jet another measure for the performance and is calculated by
$$ \mathrm{gain} = \frac{\mathrm{accuracy}}{\mathrm{random\; accuracy}} \; ,$$
where
$$ \mathrm{random\; accuracy} = \left(\frac{\mathrm{positive\; targets}}{\mathrm{\# instances}}\right)^2 +
\left(\frac{\mathrm{negative\; targets}}{\mathrm{\# instances}}\right)^2 \; .$$
In this setup a gain equal to one would indicate no better classification than random and
it is true that the higher the gain, the better the classifier.
It is useful for comparison of different classifiers given the same dataset.
Here it confirms that Bayes and Copula perform approximately equal
while the pure Density Tree is less informative due to the imbalance in splits mentioned
in Section~\ref{res}.
\newline

There are evidently two types of classification errors that can occur, a word that gets classified
as a nonword (type I error) and the opposite, a nonword classified as a valid word (type II error).
Analysis on this error behaviour is shown in Table~\ref{tab8}.

\begin{table}[h]
	\centering
	\begin{tabular}{l c c c }
		wordlength	& Bayes		& Copula	& Density Tree	\\
		5			&	1.200	& 0.946		& 4.933			\\
	 	7			&   1.116	& 1.143		& 8.249			\\
		10 			& 	1.127	& 1.192		& 38.55			\\
	\end{tabular}
	\caption{Error analysis (type I error/type II error) for different classifiers.}
	\label{tab8}
\end{table}

What is striking is the high occurrence of type I errors for the Density Tree.
It explains the low classification rate and states that a lot of valid words are classified as nonwords.
For the Bayes and Copula the errors are balanced. This indicates that i) Bayes suppresses the type I errors
coming from the Density Tree part of the Copula or ii) that the Density Tree behaves differently
when trained on the cumulative density.
Either way the Bayes classifier seems to govern the overall behaviour of the Copula classifier.
This of course prevents Copula from neither performing worse nor better than Bayes.
\newline
Finally we will check which classifier generates the best results.

\subsubsection{Bayes}

\subsubsection{Density Tree}

\subsubsection{Copula}


\section{Summary}


\end{document}
